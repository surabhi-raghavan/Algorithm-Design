\documentclass[12pt]{article}

% Packages
\usepackage[utf8]{inputenc}
\usepackage{amsmath, amssymb}
\usepackage{graphicx}
\usepackage{fancyhdr}
\usepackage{geometry}
\usepackage{algorithm}
\usepackage{algpseudocode}
\geometry{margin=1in}

% Header/Footer Configuration
\pagestyle{fancy}
\fancyhf{}
\fancyhead[L]{\yourname}
\fancyhead[C]{Homework \#6}
\fancyhead[R]{\coursecode}
\fancyfoot[C]{\thepage}

% Custom commands
\newcommand{\yourname}{Surabhi Raghavan}  % Your name
\newcommand{\hwnumber}{6}          % Homework number
\newcommand{\coursecode}{INFSCI 2591}  % Course name/code

% Begin document
\begin{document}

\title{Homework \#6}
\author{\yourname}
\date{\today}

\maketitle

\section*{Problem 4}
% Problem description
Write an algorithm that takes an integer n as input and determines the total
number of solutions to the n-Queens problem.
\subsection*{Solution}
% Your solution goes here.
The n-queens problem  involves placing n queens on a n x n chessboard such that no two queens threaten each other.
\\
Two queens do not threaten each other if:
\begin{itemize}
\item if 2 queens are in the same row
\item if 2 queens are in the same column 
\item if 2 queens are iin the same diagonal 
\end{itemize}
We need two separate functions- one to place the queen in a particular position and another to determine all combinations of the solutions.\\
\\
Alogrithm: Determine if a queen can  be placed at a particular position \\
Input: k (int)- represents the kth queen \\
Output: Returns 1 if the queen can be placed and 0 if it cannot\\

\begin{algorithm}
    \caption{Place($k, i$)}
    \begin{algorithmic}[1]
        \For{$j = 1$ to $k-1$}
            \If{$x[j] = i$ \textbf{or} $|j - k| = |x[j] - i|$}
                \State \Return 0
            \EndIf
        \EndFor
        \State \Return 1
    \end{algorithmic}
    \end{algorithm}
    

Alogrithm: Function to determine the number of solutions to the n-queens problem\\
Input: n (int)- represents the number of queens and the size of the chessboard \\
Output: Number of solutions\\

\begin{algorithm}
    \caption{NQueens(n)}
    \begin{algorithmic}[1]
        \State $k \gets 1$
        \State $x[k] \gets 0$
        \State $solution\_count \gets 0$ \Comment{Initialize counter for number of solutions}
        \While{$k > 0$}
            \State $x[k] \gets x[k] + 1$
            \While{$x[k] \leq n$ \textbf{and} not PLACE($k, x[k]$)}
                \State $x[k] \gets x[k] + 1$
            \EndWhile
            \If{$x[k] \leq n$}
                \If{$k == n$}
                    \State $solution\_count \gets solution\_count + 1$ \Comment{Increment solution count}
                \Else
                    \State $k \gets k + 1$
                    \State $x[k] \gets 0$
                \EndIf
            \Else
                \State $k \gets k - 1$
            \EndIf
        \EndWhile
        \State \Return $solution\_count$ \Comment{Return the total number of solutions found}
    \end{algorithmic}
    \end{algorithm}
\section*{Problem 10}
% Problem description
Find at least two instances of the n-Queens problem that have no solutions.
\subsection*{Solution}


\section*{Problem }
% Problem description
State the problem here.

\subsection*{Solution}
% Your solution goes here.

\end{document}
